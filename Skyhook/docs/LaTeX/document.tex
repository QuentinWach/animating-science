\documentclass{article}
\usepackage[utf8]{inputenc}
\usepackage{graphicx}
\usepackage{amsmath}
\usepackage{amsfonts}
\graphicspath{ {./images/} }
\usepackage{geometry}
\geometry{
a4paper,
total={170mm,257mm},
left=20mm,
top=20mm}
\usepackage{graphicx}

\renewcommand{\figurename}{Abbildung}


\title{ \textbf{Der Skyhook} \\ \textmd{Eine Infrastruktur zwischen Erde und Mars}}  % \\ \textmd{}

\author{
  Quentin Wach \\
  Fabian Heisinger
  \date{2. August 2019}
}

\begin{document}
\maketitle
\begin{abstract}
Traditionelle Raumfahrt ist teuer, gefährlich und unfassbar Ressourcen-intensiv. Ein einfacherer Weg  ins All zu kommen ist ein sogenannter Skyhook oder Spacetether,  ein ständig rotierendes Seil, das Raumschiffe wie ein Katapult aus dem Orbit ins All schießt. Wir simulieren eine Skyhook Infrastruktur im Sonnensystem.
\end{abstract}

\section{Hintergrund}

\section{Umsetzung}
\subsection{Sonnensystem}
Wir betrachten lediglich Erde und Mars im Umlauf um die Sonne in der x-y-Ebende. Es werden die Keplerregeln zunächst vernachlässigt und die Bahnen der Planeten als Kreise aufgefasst. Die Abstände zur Sonne werden dabei zeitlich gemittelt. Beide Planeten werden ebenfalls als Kreise mit diskreten Radien beschrieben. Ihre Umlaufzeiten variieren wie auch ihre Radien, Massen und Abstände zur Sonne.  
\subsection{Raumfahrtzeug}
\subsection{Interplanetarischer Ballistischer Flug}
Um von einem Planeten zum nächsten zu gelangen muss die Rakete zunächst die Erde verlassen. Das Erreichen der dazu nötigen Fluchtgeschwindigkeit ist jedoch nur ein Teil des Problems zu dem außerdem das Gravitationsfeld der Sonne und die Bewegung der Erde um die Sonne betrachtet werden muss. \\

Es wird die höhere Rotationsgeschwindigkeit der Erde um die Sonne als die des Mars genutzt, um diesen - wenn nötig - einzuholen. Entscheidend ist die radiale Geschwindgikeit der Rakete.

Hohmann-Transfer nicht möglich.
\section{Diskussion}



\end{document}
